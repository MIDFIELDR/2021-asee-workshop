% packages
\usepackage[]{graphicx}
  \graphicspath{{../visuals/}}  % relative path
\usepackage{color}
  \definecolor{myviolet}{HTML}{3333B2}
\usepackage{etex}     % corrects a "new \dimen" problem
\usepackage{bibentry} % for full-text at the point of citation
\usepackage{natbib}
  \bibliographystyle{plain}
\usepackage[latin1]{inputenc} % typeface
\usepackage{times}            % typeface
\usepackage[T1]{fontenc}      % typeface
\usepackage{enumitem} % for lists
% set enumitem symbol, margins, and spacing between items
\setlist[itemize]{label=\color{myviolet}$\blacktriangleright$, itemsep=1\baselineskip, leftmargin=*, font=\large}
\usepackage{bold-extra} % for bold tt font

\usepackage{booktabs} % for tables
% \usepackage{ctable}   % for use with xtable printing

% if I need math
\usepackage{latexsym}
\usepackage{amsmath}
\usepackage{amssymb}

% a separate file for creating drop shadow under a figure
% (The script is included at the end of this file)
% then use \shadowimage[]{} instead of \includegraphics[]{}
% \input{../Design/test_drop_shadow}

% reset the beamer template
\usetheme{default}
\beamertemplatenavigationsymbolsempty % omit navigation symbols
\setbeamercolor{title}{fg=myviolet} % title slide main title text
\setbeamercolor{frametitle}{fg=myviolet} % slide headline
% \setbeamercolor{structure}{fg=red} % bullets

% this chunk needs more work, but when working I can use [12pt] in the
% document class and ersase all the \large commands in the
% individual slides AND I can use the chunk below to reduce the [12pt]
% frametitle font size (default is too big)
% \setbeamertemplate{frametitle}
% {
%     \nointerlineskip
%     \begin{beamercolorbox}[sep=3mm,ht=3.5em,wd=\paperwidth]{frametitle}
%         \vbox{}\vskip-0ex%
%         \strut\insertframetitle\strut
%         \vskip-0ex%
%         \vfill%
%     \end{beamercolorbox}
% }

% eliminate the "newblocks" (\par) in the full-text citations
\setbeamertemplate{bibliography entry author} {\color{gray}}
\setbeamertemplate{bibliography entry title}  {\color{gray}}
\setbeamertemplate{bibliography entry journal}{\color{gray}}
\setbeamertemplate{bibliography entry year}   {\color{gray}}
\setbeamertemplate{bibliography entry volume} {\color{gray}}
\setbeamertemplate{bibliography entry number} {\color{gray}}
\setbeamertemplate{bibliography entry month}  {\color{gray}}
\setbeamertemplate{bibliography entry pages}  {\color{gray}}
\setbeamertemplate{bibliography entry location}{\color{gray}}
\setbeamertemplate{bibliography entry note}   {\color{gray}}

% custom footline with note on left and page number on the right
\makeatother%
\setbeamertemplate{footline}%
{%
\leavevmode%
\hbox{\fontsize{6}{12}\selectfont% in a 10pt document, 7pt is about 16pt on the slide. the {12} linespacing option does not seem to work yet
% on the right, I have a space for permissions note or full citation
\begin{beamercolorbox}[wd=.88\paperwidth,ht=10ex,dp=3mm,leftskip=3mm,rightskip=0mm]{}\color{gray}\foottext%
\end{beamercolorbox}%
% on the left is the slide number
\begin{beamercolorbox}[wd=.12\paperwidth,ht=10ex,dp=3mm,leftskip=0mm,rightskip=3mm]{}\hfill\insertpagenumber\ / \inserttotalframenumber%
\end{beamercolorbox}%
}%
\vskip0pt%
}%
\makeatletter%

% set parameters for the PDF file
\hypersetup{
pdftitle={\talktitle},
pdfauthor={\talkauthor},
pdfkeywords={persistence,} {student record data},
pdfnewwindow=true,  % links in new PDF window
colorlinks=true,    % false: boxed links; true: colored links
linkcolor=myviolet, % internal links (change box color w/ linkbordercolor)
citecolor=myviolet, % color of links to bibliography
filecolor=myviolet, % color of file links
urlcolor=myviolet   % color of external links
}

% new commands and definitions
% headline for a slide (creates an argument for \title{})
\newcommand\headline{}

% used for text in a footer, e.g., a full citation or permissions
\newcommand\foottext{}

% a black-out slide
\newcommand{\blackoutslide}{%
{% this bracket makes the black background apply just once
\setbeamercolor{background canvas}{bg=black}
\renewcommand\headline{}
\renewcommand\foottext{}
\begin{frame}[t]
\end{frame}
}
}

% ensures that graphics do not exceed slide margins
% maxwidth is the least of original width or linewidth
\makeatletter
\def\maxwidth{ %
  \ifdim\Gin@nat@width>\linewidth
    \linewidth
  \else
    \Gin@nat@width
  \fi
}
\makeatother

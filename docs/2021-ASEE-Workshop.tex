% Options for packages loaded elsewhere
\PassOptionsToPackage{unicode}{hyperref}
\PassOptionsToPackage{hyphens}{url}
%
\documentclass[
]{book}
\title{Engaging with MIDFIELD Data}
\author{Susan Lord, Matthew Ohland, Marisa Orr, Richard Layton, and Russell Long}
\date{2021-07-26}

\usepackage{amsmath,amssymb}
\usepackage{lmodern}
\usepackage{iftex}
\ifPDFTeX
  \usepackage[T1]{fontenc}
  \usepackage[utf8]{inputenc}
  \usepackage{textcomp} % provide euro and other symbols
\else % if luatex or xetex
  \usepackage{unicode-math}
  \defaultfontfeatures{Scale=MatchLowercase}
  \defaultfontfeatures[\rmfamily]{Ligatures=TeX,Scale=1}
\fi
% Use upquote if available, for straight quotes in verbatim environments
\IfFileExists{upquote.sty}{\usepackage{upquote}}{}
\IfFileExists{microtype.sty}{% use microtype if available
  \usepackage[]{microtype}
  \UseMicrotypeSet[protrusion]{basicmath} % disable protrusion for tt fonts
}{}
\makeatletter
\@ifundefined{KOMAClassName}{% if non-KOMA class
  \IfFileExists{parskip.sty}{%
    \usepackage{parskip}
  }{% else
    \setlength{\parindent}{0pt}
    \setlength{\parskip}{6pt plus 2pt minus 1pt}}
}{% if KOMA class
  \KOMAoptions{parskip=half}}
\makeatother
\usepackage{xcolor}
\IfFileExists{xurl.sty}{\usepackage{xurl}}{} % add URL line breaks if available
\IfFileExists{bookmark.sty}{\usepackage{bookmark}}{\usepackage{hyperref}}
\hypersetup{
  pdftitle={Engaging with MIDFIELD Data},
  pdfauthor={Susan Lord, Matthew Ohland, Marisa Orr, Richard Layton, and Russell Long},
  hidelinks,
  pdfcreator={LaTeX via pandoc}}
\urlstyle{same} % disable monospaced font for URLs
\usepackage{color}
\usepackage{fancyvrb}
\newcommand{\VerbBar}{|}
\newcommand{\VERB}{\Verb[commandchars=\\\{\}]}
\DefineVerbatimEnvironment{Highlighting}{Verbatim}{commandchars=\\\{\}}
% Add ',fontsize=\small' for more characters per line
\usepackage{framed}
\definecolor{shadecolor}{RGB}{248,248,248}
\newenvironment{Shaded}{\begin{snugshade}}{\end{snugshade}}
\newcommand{\AlertTok}[1]{\textcolor[rgb]{0.94,0.16,0.16}{#1}}
\newcommand{\AnnotationTok}[1]{\textcolor[rgb]{0.56,0.35,0.01}{\textbf{\textit{#1}}}}
\newcommand{\AttributeTok}[1]{\textcolor[rgb]{0.77,0.63,0.00}{#1}}
\newcommand{\BaseNTok}[1]{\textcolor[rgb]{0.00,0.00,0.81}{#1}}
\newcommand{\BuiltInTok}[1]{#1}
\newcommand{\CharTok}[1]{\textcolor[rgb]{0.31,0.60,0.02}{#1}}
\newcommand{\CommentTok}[1]{\textcolor[rgb]{0.56,0.35,0.01}{\textit{#1}}}
\newcommand{\CommentVarTok}[1]{\textcolor[rgb]{0.56,0.35,0.01}{\textbf{\textit{#1}}}}
\newcommand{\ConstantTok}[1]{\textcolor[rgb]{0.00,0.00,0.00}{#1}}
\newcommand{\ControlFlowTok}[1]{\textcolor[rgb]{0.13,0.29,0.53}{\textbf{#1}}}
\newcommand{\DataTypeTok}[1]{\textcolor[rgb]{0.13,0.29,0.53}{#1}}
\newcommand{\DecValTok}[1]{\textcolor[rgb]{0.00,0.00,0.81}{#1}}
\newcommand{\DocumentationTok}[1]{\textcolor[rgb]{0.56,0.35,0.01}{\textbf{\textit{#1}}}}
\newcommand{\ErrorTok}[1]{\textcolor[rgb]{0.64,0.00,0.00}{\textbf{#1}}}
\newcommand{\ExtensionTok}[1]{#1}
\newcommand{\FloatTok}[1]{\textcolor[rgb]{0.00,0.00,0.81}{#1}}
\newcommand{\FunctionTok}[1]{\textcolor[rgb]{0.00,0.00,0.00}{#1}}
\newcommand{\ImportTok}[1]{#1}
\newcommand{\InformationTok}[1]{\textcolor[rgb]{0.56,0.35,0.01}{\textbf{\textit{#1}}}}
\newcommand{\KeywordTok}[1]{\textcolor[rgb]{0.13,0.29,0.53}{\textbf{#1}}}
\newcommand{\NormalTok}[1]{#1}
\newcommand{\OperatorTok}[1]{\textcolor[rgb]{0.81,0.36,0.00}{\textbf{#1}}}
\newcommand{\OtherTok}[1]{\textcolor[rgb]{0.56,0.35,0.01}{#1}}
\newcommand{\PreprocessorTok}[1]{\textcolor[rgb]{0.56,0.35,0.01}{\textit{#1}}}
\newcommand{\RegionMarkerTok}[1]{#1}
\newcommand{\SpecialCharTok}[1]{\textcolor[rgb]{0.00,0.00,0.00}{#1}}
\newcommand{\SpecialStringTok}[1]{\textcolor[rgb]{0.31,0.60,0.02}{#1}}
\newcommand{\StringTok}[1]{\textcolor[rgb]{0.31,0.60,0.02}{#1}}
\newcommand{\VariableTok}[1]{\textcolor[rgb]{0.00,0.00,0.00}{#1}}
\newcommand{\VerbatimStringTok}[1]{\textcolor[rgb]{0.31,0.60,0.02}{#1}}
\newcommand{\WarningTok}[1]{\textcolor[rgb]{0.56,0.35,0.01}{\textbf{\textit{#1}}}}
\usepackage{longtable,booktabs,array}
\usepackage{calc} % for calculating minipage widths
% Correct order of tables after \paragraph or \subparagraph
\usepackage{etoolbox}
\makeatletter
\patchcmd\longtable{\par}{\if@noskipsec\mbox{}\fi\par}{}{}
\makeatother
% Allow footnotes in longtable head/foot
\IfFileExists{footnotehyper.sty}{\usepackage{footnotehyper}}{\usepackage{footnote}}
\makesavenoteenv{longtable}
\usepackage{graphicx}
\makeatletter
\def\maxwidth{\ifdim\Gin@nat@width>\linewidth\linewidth\else\Gin@nat@width\fi}
\def\maxheight{\ifdim\Gin@nat@height>\textheight\textheight\else\Gin@nat@height\fi}
\makeatother
% Scale images if necessary, so that they will not overflow the page
% margins by default, and it is still possible to overwrite the defaults
% using explicit options in \includegraphics[width, height, ...]{}
\setkeys{Gin}{width=\maxwidth,height=\maxheight,keepaspectratio}
% Set default figure placement to htbp
\makeatletter
\def\fps@figure{htbp}
\makeatother
\setlength{\emergencystretch}{3em} % prevent overfull lines
\providecommand{\tightlist}{%
  \setlength{\itemsep}{0pt}\setlength{\parskip}{0pt}}
\setcounter{secnumdepth}{5}
\usepackage{booktabs}
\usepackage{amsthm}
\makeatletter
\def\thm@space@setup{%
  \thm@preskip=8pt plus 2pt minus 4pt
  \thm@postskip=\thm@preskip
}
\makeatother
\ifLuaTeX
  \usepackage{selnolig}  % disable illegal ligatures
\fi
\usepackage[]{natbib}
\bibliographystyle{plainnat}

\begin{document}
\maketitle

{
\setcounter{tocdepth}{1}
\tableofcontents
}
\hypertarget{introduction}{%
\chapter{Introduction}\label{introduction}}

\hypertarget{objectives}{%
\section{Objectives}\label{objectives}}

MIDFIELD---The \emph{Multiple-Institution Database for Investigating Engineering Longitudinal Development}---is a partnership of higher education institutions with engineering programs. MIDFIELD contains student record data from 1988--2017 for approximately one million undergraduate, degree-seeking students at the partner institutions.

The goal of this workshop is to make MIDFIELD more accessible to the ASEE community. The workshop introduces \emph{midfieldr} (a package in the R software environment) that provides access to a MIDFIELD student-record data sample and tools to analyze and graph persistence metrics such as graduation rates. The workshop is designed for R beginners.

By the end of the workshop, participants should be able to:

\begin{itemize}
\tightlist
\item
  Describe key variables in MIDFIELD data tables
\item
  Explore and tell a story from MIDFIELD data
\item
  Use midfieldr, an R package specifically designed for use with MIDFIELD
\item
  Explain key features of effective data displays
\end{itemize}

\hypertarget{description}{%
\section{Description}\label{description}}

The robustness of the MIDFIELD data allows us to emphasize an intersectional approach to the study of student records, permitting multiple categories of inequity such as race/ethnicity and sex to be considered simultaneously.

To introduce beginners to R, participants work through a self-paced tutorial covering basic elements of the R computing language and environment. To introduce midfieldr and using it to work with student record data, participants work through a ``Get started'' tutorial in which they determine the numbers of students ever enrolled in two programs, group and summarize the data, and graph the results.

For more experienced R users or anyone working at a faster pace, we offer a series of self-paced tutorials that introduce key features of midfieldr and how they are applied to compute persistence metrics and graph results.

We also discuss the merits of the multiway graph design that is recommended for displaying results of this type. The agenda includes an interactive session to demonstrate contemporary principles of effective data display.

\hypertarget{pre-workshop-homework}{%
\section{Pre-workshop homework}\label{pre-workshop-homework}}

To get the most out of the workshop, you should have the essential software installed and running several days before the workshop to give you time to contact us with questions if anything goes amiss.

Your homework is explained on the pages:

\begin{itemize}
\tightlist
\item
  \protect\hyperlink{install-everything}{Install everything}
\item
  {[}Setup a project{]}
\end{itemize}

\hypertarget{agenda}{%
\section{Agenda}\label{agenda}}

Our three hours are organized approximately as shown.

\begin{tabular}{ll}
\toprule
Min & Topic\\
\midrule
15 & Introduction\\
35 & Exploring the data structure\\
35 & Working with R\\
15 & Break\\
20 & Designing effective displays\\
\addlinespace
50 & Working with R (continued)\\
10 & Next steps \& assessing the workshop\\
\bottomrule
\end{tabular}

\hypertarget{facilitators}{%
\section{Facilitators}\label{facilitators}}

\begin{description}
\item[Susan Lord]
Director of the MIDFIELD Institute and Professor and Chair of Integrated Engineering at the University of San Diego. She is a Fellow of the IEEE and the ASEE. Dr.~Lord has considerable experience facilitating workshops including the National Effective Teaching Institute (NETI) and special sessions at FIE. (\href{mailto:slord@sandiego.edu}{\nolinkurl{slord@sandiego.edu}})
\item[Matthew Ohland]
MIDFIELD Director and Principal Investigator. He is Professor and Associate Head of Engineering Education at Purdue University and a Fellow of IEEE, ASEE, and AAAS. Dr.~Ohland has considerable experience facilitating workshops including the NETI and CATME training. (\href{mailto:ohland@purdue.edu}{\nolinkurl{ohland@purdue.edu}})
\item[Marisa Orr]
MIDFIELD Associate Director and Associate Professor in Engineering and Science Education with a joint appointment in Mechanical Engineering at Clemson University. She received the 2009 Helen Plants Award for the best nontraditional session at FIE, ``Enhancing Student Learning Using SCALE-UP Format.'' (\href{mailto:marisak@clemson.edu}{\nolinkurl{marisak@clemson.edu}})
\item[Richard Layton]
MIDFIELD Data Visualization Specialist and Professor Emeritus of Mechanical Engineering at Rose-Hulman Institute of Technology. He is the lead developer of the R packages used in this workshop. Dr.~Layton has considerable experience facilitating workshops, including FIE workshops on data visualization (2014) and midfieldr (2018). (\href{mailto:graphdoctor@gmail.com}{\nolinkurl{graphdoctor@gmail.com}})
\item[Russell Long]
MIDFIELD Managing Director and Data Steward. He developed the stratified data sample for the R packages used in this workshop. Mr.~Long is a SAS expert with over twenty years of experience in institutional research and assessment. (\href{mailto:ralong@purdue.edu}{\nolinkurl{ralong@purdue.edu}})
\end{description}

\hypertarget{licenses}{%
\section{Licenses}\label{licenses}}

The following licenses apply to the text, data, and code in these workshops. Our goal is to minimize legal encumbrances to the dissemination, sharing, use, and re-use of this work. However, the existing rights of authors whose work is cited (text, code, or data) are reserved to those authors.

\begin{itemize}
\tightlist
\item
  \href{https://creativecommons.org/licenses/by/4.0/legalcode}{CC-BY 4.0} for all text\\
\item
  \href{https://www.r-project.org/Licenses/GPL-3}{GPL-3} for all code\\
\item
  \href{https://wiki.creativecommons.org/wiki/CC0_use_for_data}{CC0} for all data
\end{itemize}

\hypertarget{acknowledgement}{%
\section{Acknowledgement}\label{acknowledgement}}

Funding provided by the National Science Foundation Grant 1545667 ``Expanding Access to and Participation in the Multiple-Institution Database for Investigating Engineering Longitudinal Development.''

\protect\hyperlink{introduction}{▲ top of page}

\hypertarget{install-everything}{%
\chapter{Install everything}\label{install-everything}}

If you are trying R for the first time, it is vital that you attempt to set up your computer with the necessary software in advance or it will be difficult to keep up.

Your pre-workshop homework:

\begin{itemize}
\tightlist
\item
  \protect\hyperlink{install-r-and-rstudio}{Install R and RStudio}
\item
  \protect\hyperlink{create-a-project}{Create a project}\\
\item
  \protect\hyperlink{add-some-folders}{Add some folders}
\item
  \protect\hyperlink{create-a-library-for-packages}{Create a library for packages}
\item
  \protect\hyperlink{create-the-.renviron-file}{Create the .Renviron file}
\item
  \protect\hyperlink{install-packages-we-need}{Install packages we need}
\item
  \protect\hyperlink{install-midfieldr}{Install midfieldr}
\item
  \protect\hyperlink{install-midfielddata}{Install midfielddata}
\end{itemize}

If you already have R and RStudio installed, this is a great time to

\begin{itemize}
\tightlist
\item
  Make sure your R installation is current
\item
  Make sure your RStudio installation is current
\item
  Update your packages by running:
\end{itemize}

\begin{Shaded}
\begin{Highlighting}[]
\FunctionTok{update.packages}\NormalTok{(}\AttributeTok{ask =} \ConstantTok{FALSE}\NormalTok{, }\AttributeTok{checkBuilt =} \ConstantTok{TRUE}\NormalTok{)}
\end{Highlighting}
\end{Shaded}

\hypertarget{install-r-and-rstudio}{%
\section{Install R and RStudio}\label{install-r-and-rstudio}}

The first steps are to install R and RStudio. Windows users may have to login as an Administrator before installing the software.

\begin{itemize}
\tightlist
\item
  \href{https://cloud.r-project.org}{Install R} for your operating system\\
\item
  \href{https://www.rstudio.com/products/rstudio/\#Desktop}{Install RStudio}, a user interface for R
\end{itemize}

Once the installation is complete, you can take a 2-minute tour of the RStudio interface.

\begin{itemize}
\tightlist
\item
  \href{https://www.youtube.com/embed/kfcX5DEMAp4?start=57\&end=152}{Let's start (00:57--02:32)} by R Ladies Sydney \citep{RLadiesSydney:2018:Lesson1}
\end{itemize}

The same video includes a longer (7 minute) tour of the four quadrants (panes) in RStudio if you are interested.

\begin{itemize}
\tightlist
\item
  \href{https://www.youtube.com/embed/kfcX5DEMAp4?start=441\&end=880}{The RStudio quadrants (07:21--14:40)} by R Ladies Sydney \citep{RLadiesSydney:2018:Lesson1}
\end{itemize}

\hypertarget{create-a-project}{%
\section{Create a project}\label{create-a-project}}

To begin any project, we create an RStudio \emph{Project} file and directory. You can recognize an R project file by its \emph{.Rproj} suffix.

If you prefer your instructions with commentary,

\begin{itemize}
\tightlist
\item
  \href{https://www.youtube.com/embed/kfcX5DEMAp4?start=154\&end=290}{Start with a Project (02:34--04:50)} by R Ladies Sydney \citep{RLadiesSydney:2018:Lesson1}
\end{itemize}

If you prefer basic written instructions,

\begin{itemize}
\tightlist
\item
  RStudio, \emph{File \textgreater{} New Project\ldots{} \textgreater{} New Directory \textgreater{} New Project}
\item
  Or, click the \emph{New Project} button in the Console ribbon,
\end{itemize}

In the dialog box that appears,

\begin{itemize}
\tightlist
\item
  Type the workshop name as the directory name, for example, \texttt{workshop}, or if you like more detail, \texttt{midfield-workshop-asee-2021}\\
\item
  Use the browse button to select a location on your computer to create the project folder\\
\item
  Click the \emph{Create Project} button
\end{itemize}

Whenever you work with the workshop materials, launch the \texttt{workshop.Rproj} file (using the name you actually used) to start the session.

\hypertarget{add-some-folders}{%
\section{Add some folders}\label{add-some-folders}}

While file organization is a matter of personal preference, we ask that you use the directory structure shown here for your work in the workshop. Assuming we called our project \texttt{workshop}, the minimal directory structure has three folders in it plus the \texttt{.Rproj} file at the top level.

\begin{Shaded}
\begin{Highlighting}[]
\NormalTok{\textbackslash{}workshop}
\NormalTok{    \textbackslash{}data}
\NormalTok{    \textbackslash{}results}
\NormalTok{    \textbackslash{}scripts}
\NormalTok{    workshop.Rproj}
\end{Highlighting}
\end{Shaded}

We use the folders as follows:

\begin{itemize}
\tightlist
\item
  \texttt{data} data files
\item
  \texttt{results} finished graphs and tabulated data formatted for display\\
\item
  \texttt{scripts} R scripts that operate on data to produce results
\end{itemize}

If you prefer your instructions with commentary,

\begin{itemize}
\tightlist
\item
  \href{https://www.youtube.com/embed/kfcX5DEMAp4?start=290\&end=368}{Make some folders (04:50--06:08)} by R Ladies Sydney \citep{RLadiesSydney:2018:Lesson1}
\end{itemize}

If you prefer basic written instructions,

\begin{itemize}
\tightlist
\item
  use your usual method of creating new folders on your machine
\item
  or you can use the \emph{New Folder} button in the Files pane
\end{itemize}

\hypertarget{create-a-library-for-packages}{%
\section{Create a library for packages}\label{create-a-library-for-packages}}

Packages are like ``apps'' for R. As Hadley Wickham states, ``Packages are the fundamental units
of reproducible R code. They include reusable functions, the
documentation that describes how to use them, and sample data.''

If we store packages in a library separate from the base R installation,
then when you update R, you don't have to reinstall every package,
saving a lot of time.

If you happen to be working with R on a network, storing your library locally can save you time tracking down obscure problems that R has with network drives as well.

At the top level of your drive, create a new directory (folder) named
``R''. In that directory create a new folder named ``library'', for
example,

\begin{itemize}
\tightlist
\item
  Windows: \texttt{C:/R/library}\\
\item
  Mac OS and Linux: \texttt{\textasciitilde{}/R/library}
\end{itemize}

\hypertarget{create-the-.renviron-file}{%
\section{Create the .Renviron file}\label{create-the-.renviron-file}}

The \texttt{.Renviron} file is a text file that directs R packages to be
installed in the \texttt{R/library} directory created earlier. To create the
file, with RStudio open,

\begin{itemize}
\tightlist
\item
  Create a new text file, \emph{File menu \textgreater{} New File \textgreater{} Text File}.
\item
  Save the file to the \texttt{workshop} main directory using the filename
  \emph{.Renviron}
\end{itemize}

In this file, write the following line of text that tells R the path to
the stand-alone package library you created earlier.

\begin{itemize}
\tightlist
\item
  Windows: \texttt{R\_LIBS\_USER="C:/R/library"}
\item
  Mac OS and Linux: \texttt{R\_LIBS\_USER="\textasciitilde{}/R/library"}
\end{itemize}

Save and close to recognize the .Renviron file.

\begin{itemize}
\tightlist
\item
  Save and close the \texttt{.Renviron} file.
\item
  Close RStudio
\end{itemize}

Let's check your project directory. You should have at least the
following folders and files,

\begin{Shaded}
\begin{Highlighting}[]
\NormalTok{\textbackslash{}workshop}
\NormalTok{    \textbackslash{}data}
\NormalTok{    \textbackslash{}results}
\NormalTok{    \textbackslash{}scripts}
\NormalTok{    .Renviron}
\NormalTok{    workshop.Rproj}
\end{Highlighting}
\end{Shaded}

Remember, every time you create a new project, paste a copy of the \texttt{.Renviron} file at the top level of the project directory.

\hypertarget{install-packages-we-need}{%
\section{Install packages we need}\label{install-packages-we-need}}

The fundamental unit of shareable code in R is the \emph{package.} For the R novice, an R package is like an ``app'' for R---a collection of functions, data, and documentation for doing work in R that is easily shared with others \citep{wickham2014advanced}.

Most packages are obtained from the \href{https://cran.r-project.org/}{CRAN} website \citep{cranweb}. To install a CRAN package using RStudio:

\begin{itemize}
\tightlist
\item
  Launch RStudio
\end{itemize}

The RStudio interface has several panes. We want the \emph{Files/Plots/Packages} pane.

\begin{itemize}
\tightlist
\item
  Select the \emph{Packages} tab
\end{itemize}

Next,

\begin{itemize}
\tightlist
\item
  Click \emph{Install} on the ribbon
\item
  In the dialog box, type the name of the package. For our first package, type \texttt{data.table} to install the data.table package \citep{R-data.table}
\item
  Check the \emph{Install dependencies} box
\item
  Click the \emph{Install} button
\end{itemize}

\begin{quote}
During the installation, Windows users might get a warning message about
Rtools, something like:

\texttt{WARNING:\ Rtools\ is\ required\ to\ build\ R\ packages\ but\ is\ not\ currently\ installed.\ Please\ download\ and\ install\ the\ appropriate\ version...}.

You may ignore the warning and carry on.
\end{quote}

In the RStudio Console, you should see a message like this one,

\begin{Shaded}
\begin{Highlighting}[]
\NormalTok{    package }\StringTok{\textquotesingle{}data.table\textquotesingle{}}\NormalTok{ successfully unpacked and MD5 sums checked}
\end{Highlighting}
\end{Shaded}

If successful, the package will appear in the Packages pane, e.g.,

Repeat the process for the following packages

\begin{verbatim}
wrapr 
Rdpack 
checkmate
\end{verbatim}

Alternatively, you can install them all at once by typing in the Console:

\begin{Shaded}
\begin{Highlighting}[]
\CommentTok{\# midfieldr depends on these packages}
\NormalTok{packages\_we\_need }\OtherTok{\textless{}{-}} \FunctionTok{c}\NormalTok{(}\StringTok{"data.table"}\NormalTok{, }\StringTok{"wrapr"}\NormalTok{, }\StringTok{"Rdpack"}\NormalTok{, }\StringTok{"checkmate"}\NormalTok{)}
\FunctionTok{install.packages}\NormalTok{(packages\_we\_need)}
\end{Highlighting}
\end{Shaded}

\hypertarget{install-midfieldr}{%
\section{Install midfieldr}\label{install-midfieldr}}

midfieldr is not yet available from \href{https://cran.r-project.org/}{CRAN}. To install the development version of midfieldr from its \texttt{drat} repository, type in the Console:

\begin{Shaded}
\begin{Highlighting}[]
\CommentTok{\# install midfieldr from drat repo}
\FunctionTok{install.packages}\NormalTok{(}\StringTok{"midfieldr"}\NormalTok{, }
                 \AttributeTok{repos =} \StringTok{"https://MIDFIELDR.github.io/drat/"}\NormalTok{, }
                 \AttributeTok{type =} \StringTok{"source"}\NormalTok{)}
\end{Highlighting}
\end{Shaded}

You can confirm a successful installation by running the following lines to bring up the package help page in the Help window.

\begin{Shaded}
\begin{Highlighting}[]
\FunctionTok{library}\NormalTok{(}\StringTok{"midfieldr"}\NormalTok{)}
\FunctionTok{help}\NormalTok{(}\StringTok{"midfieldr{-}package"}\NormalTok{)}
\end{Highlighting}
\end{Shaded}

If the installation is successful, the code chunk above should produce a view of the help page as shown here.

\hypertarget{install-midfielddata}{%
\section{Install midfielddata}\label{install-midfielddata}}

Because of its size, the data package is stored in a \texttt{drat} repository instead of CRAN. Installation takes time; please be patient and wait for the Console prompt ``\textgreater{}'' to reappear.

Type (or copy and paste) the following lines in the RStudio Console.

\begin{Shaded}
\begin{Highlighting}[]
\CommentTok{\# type in the RStudio Console  }
\FunctionTok{install.packages}\NormalTok{(}\StringTok{"midfielddata"}\NormalTok{, }
                 \AttributeTok{repos =} \StringTok{"https://MIDFIELDR.github.io/drat/"}\NormalTok{, }
                 \AttributeTok{type =} \StringTok{"source"}\NormalTok{)}
\CommentTok{\# be patient}
\end{Highlighting}
\end{Shaded}

Once the Console prompt ``\textgreater{}'' reappears, you can confirm a successful installation by viewing the package help page. In the Console, run:

\begin{Shaded}
\begin{Highlighting}[]
\CommentTok{\# type in the RStudio Console  }
\FunctionTok{library}\NormalTok{(}\StringTok{"midfielddata"}\NormalTok{)}
\FunctionTok{help}\NormalTok{(}\StringTok{"midfielddata{-}package"}\NormalTok{)}
\end{Highlighting}
\end{Shaded}

If the installation is successful, the code chunk above should produce a view of the help page as shown here.

You finished your homework!

\protect\hyperlink{install-everything}{▲ top of page}

\hypertarget{materials}{%
\chapter{Workshop materials}\label{materials}}

\hypertarget{stuff}{%
\section{Stuff}\label{stuff}}

\hypertarget{more-stuff}{%
\section{More stuff}\label{more-stuff}}

\protect\hyperlink{materials}{▲ top of page}

\hypertarget{stories}{%
\chapter{Exploring the data structure}\label{stories}}

\hypertarget{stuff-1}{%
\section{Stuff}\label{stuff-1}}

\hypertarget{more-stuff-1}{%
\section{More stuff}\label{more-stuff-1}}

\protect\hyperlink{stories}{▲ top of page}

\hypertarget{work-with-R}{%
\chapter{Working with R}\label{work-with-R}}

R is an open source language and environment for statistical computing and graphics \citep{R-base}, ranked by IEEE in 2020 as the 6th most popular programming language (Python, Java, and C are the top three) \citep{Cass:2020}. If you are new to R, some of its best features, paraphrasing Wickham \citeyearpar{wickham2014advanced}, are:

\begin{itemize}
\tightlist
\item
  R is free, open source, and available on every major platform.
\item
  R packages provide effective tools for data analysis and visualization.
\item
  More than 17,750 open-source R packages are available (Jun 2021). Many are cutting-edge tools.
\end{itemize}

RStudio, an integrated development environment (IDE) for R, includes a console, editor, and tools for plotting, history, debugging, and workspace management as well as access to GitHub for collaboration and version control \citep{2016rstudio}.

\hypertarget{prerequisites}{%
\section{Prerequisites}\label{prerequisites}}

Before proceeding, you should have completed

\begin{itemize}
\tightlist
\item
  \protect\hyperlink{install-everything}{Install everything}\\
\item
  {[}Set up a project{]}\\
\item
  Launched your workshop project---\texttt{workshop.Rproj} or other name that you selected---to start the R session\\
\item
  We suggest you start a new R script for each tutorial and save it to the \texttt{scripts} directory. For example, at the end of the workshop, your scripts directory might contain the following files:
\end{itemize}

\begin{Shaded}
\begin{Highlighting}[]
\NormalTok{        \textbackslash{}scripts    }
\NormalTok{            \textbackslash{}R}\SpecialCharTok{{-}}\NormalTok{basics.R    }
\NormalTok{            \textbackslash{}getting}\SpecialCharTok{{-}}\NormalTok{started.R    }
\NormalTok{            \textbackslash{}case}\SpecialCharTok{{-}}\NormalTok{study}\SpecialCharTok{{-}}\NormalTok{programs.R    }
\NormalTok{            \textbackslash{}case}\SpecialCharTok{{-}}\NormalTok{study}\SpecialCharTok{{-}}\NormalTok{students.R     }
\NormalTok{            etc.  }
\end{Highlighting}
\end{Shaded}

\hypertarget{new-to-r}{%
\section{New to R?}\label{new-to-r}}

\protect\hyperlink{prerequisites}{Prerequisites} should be completed before proceeding. By the end of the workshop, our R beginners will have made progress on two or possibly three tutorials:

\begin{itemize}
\tightlist
\item
  \protect\hyperlink{r-basics}{R basics} An introduction to R, generally less than an hour.
\item
  \href{https://midfieldr.github.io/midfieldr/articles/art-000-getting-started.html}{Getting started}: An introduction to the MIDFIELD practice data tables.
\item
  \href{https://midfieldr.github.io/midfieldr/articles/art-110-case-study-programs.html}{Case study programs} Construct a data frame of program CIP codes and program names for four engineering programs (Civil, Electrical, Industrial, and Mechanical)
\end{itemize}

If you complete these tutorials and there is still time remaining, please consider moving on the to tutorials listed in the \protect\hyperlink{after-the-workshop}{After the workshop} section.

\hypertarget{familiar-with-r}{%
\section{Familiar with R?}\label{familiar-with-r}}

\protect\hyperlink{prerequisites}{Prerequisites} should be completed before proceeding. By the end of the workshop, our more experienced R users will have made substantive progress on two or possibly three tutorials:

\begin{itemize}
\tightlist
\item
  \href{https://midfieldr.github.io/midfieldr/articles/art-000-getting-started.html}{Getting started}: An introduction to the MIDFIELD practice data tables\\
\item
  \href{https://midfieldr.github.io/midfieldr/articles/art-110-case-study-programs.html}{Case study programs} Construct a data frame of program CIP codes and program names for four engineering programs (Civil, Electrical, Industrial, and Mechanical)
\item
  \href{https://midfieldr.github.io/midfieldr/articles/art-120-case-study-students.html}{Case study students} Develop a data frame of the case study students who pass the data sufficiency criterion.
\end{itemize}

If you complete these tutorials and there is still time remaining, please consider moving on the to tutorials listed in the \protect\hyperlink{after-the-workshop}{After the workshop} section.

\hypertarget{after-the-workshop}{%
\section{After the workshop}\label{after-the-workshop}}

At his point, the learning is self-directed. Choose the skills you want to continue working on. We have a set of tutorials for

\begin{itemize}
\tightlist
\item
  \protect\hyperlink{developing-r-skills}{Developing R skills}
\item
  \protect\hyperlink{continuing-the-case-study}{Continuing the case study}
\item
  \protect\hyperlink{exploring-midfieldr-functions}{Exploring midfieldr functions}
\end{itemize}

\hypertarget{developing-r-skills}{%
\subsection{Developing R skills}\label{developing-r-skills}}

The basic skills tutorials take about 50 minutes each.

\begin{itemize}
\tightlist
\item
  \protect\hyperlink{r-basics}{R basics}
\item
  \protect\hyperlink{graph-basics}{Graph basics}\\
\item
  \protect\hyperlink{data-basics}{Data basics}
\end{itemize}

\hypertarget{continuing-the-case-study}{%
\subsection{Continuing the case study}\label{continuing-the-case-study}}

The case study is a quick tour of a typical workflow using student unit record data. This is a ``big picture'' development---functions are used without detailed explanations or development so that we can get to the results with as little distraction as possible. Anyone wanting more detail will find it in the detailed vignettes (links below in \protect\hyperlink{exploring-midfieldr-functions}{Exploring midfieldr functions}).

\begin{itemize}
\tightlist
\item
  \href{https://midfieldr.github.io/midfieldr/articles/art-110-case-study-programs.html}{Case study programs}
\item
  \href{https://midfieldr.github.io/midfieldr/articles/art-120-case-study-students.html}{Case study students}
\item
  \href{https://midfieldr.github.io/midfieldr/articles/art-130-case-study-stickiness.html}{Case study stickiness}
\item
  \href{https://midfieldr.github.io/midfieldr/articles/art-140-case-study-grad-rate.html}{Case study graduation rate}
\end{itemize}

\hypertarget{exploring-midfieldr-functions}{%
\subsection{Exploring midfieldr functions}\label{exploring-midfieldr-functions}}

Deep dive into the midfieldr functionality. The work flow follows the same general pattern as the quicker case study, but pauses to explore each function in more detail, exploring the arguments and strategies for use. In general, each tutorial is self-contained so you may enter at almost any point.

\begin{itemize}
\tightlist
\item
  \href{https://midfieldr.github.io/midfieldr/articles/art-010-program-codes.html}{Program codes and names} Practice strategies of searching \texttt{cip} for programs we want to study.
\item
  \href{https://midfieldr.github.io/midfieldr/articles/art-015-subsetting-midfield-data.html}{Subsetting MIDFIELD data} Use programs codes to subset the MIDFIELD data tables.
\item
  \href{https://midfieldr.github.io/midfieldr/articles/art-020-data-sufficiency.html}{Data sufficiency} What it is and how it is applied to student unit-record (SUR) data.
\item
  \href{https://midfieldr.github.io/midfieldr/articles/art-030-timely-completion.html}{Timely completion} What it is and how it is applied to SUR data.
\item
  \href{https://midfieldr.github.io/midfieldr/articles/art-040-fye-programs.html}{FYE programs} What they are and how they are accommodated with SUR data.
\item
  \href{https://midfieldr.github.io/midfieldr/articles/art-050-multiway-graphs.html}{Multiway graphs} How to graph and interpret a common data structure encountered when working with SUR data.
\item
  \href{https://midfieldr.github.io/midfieldr/articles/art-060-tabulating-data.html}{Tabulating data} How to tabulate multiway data for publication.
\end{itemize}

\hypertarget{r-basics}{%
\section{R basics}\label{r-basics}}

This tutorial is an introduction to R adapted from \citep{Healy:2019:Ch.2} with extra material from \citep{Matloff:2019}. If you already have R experience, you might still want to browse this section in case you find something new.

\protect\hyperlink{prerequisites}{Prerequisites} should be completed before proceeding. After that, the tutorial should take no longer than 50 minutes.

\hypertarget{style-guide}{%
\subsection{Style guide}\label{style-guide}}

\hypertarget{everything-in-r-has-a-name}{%
\subsection{Everything in R has a name}\label{everything-in-r-has-a-name}}

\hypertarget{everything-in-r-is-an-object}{%
\subsection{Everything in R is an object}\label{everything-in-r-is-an-object}}

\hypertarget{use-functions-to-do-things}{%
\subsection{Use functions to do things}\label{use-functions-to-do-things}}

\hypertarget{r-functions-come-in-packages}{%
\subsection{R functions come in packages}\label{r-functions-come-in-packages}}

\hypertarget{r-objects-have-class}{%
\subsection{R objects have class}\label{r-objects-have-class}}

\hypertarget{r-objects-have-structure}{%
\subsection{R objects have structure}\label{r-objects-have-structure}}

\hypertarget{r-only-does-what-you-tell-it}{%
\subsection{R only does what you tell it}\label{r-only-does-what-you-tell-it}}

\hypertarget{keyboard-shortcuts}{%
\subsection{Keyboard shortcuts}\label{keyboard-shortcuts}}

\hypertarget{whats-next}{%
\subsection{What's next?}\label{whats-next}}

\hypertarget{graph-basics}{%
\section{Graph basics}\label{graph-basics}}

\hypertarget{data-basics}{%
\section{Data basics}\label{data-basics}}

\protect\hyperlink{start-with-R}{▲ top of page}

\hypertarget{display-design}{%
\chapter{Designing effective displays}\label{display-design}}

\hypertarget{stuff-2}{%
\section{Stuff}\label{stuff-2}}

\hypertarget{more-stuff-2}{%
\section{More stuff}\label{more-stuff-2}}

\protect\hyperlink{display-design}{▲ top of page}

  \bibliography{book.bib,packages.bib,references.bib}

\end{document}
